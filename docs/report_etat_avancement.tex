\documentclass{article}
\usepackage[utf8]{inputenc}
\usepackage[T1]{fontenc}
\usepackage[french]{babel}
\usepackage{graphicx}
\usepackage{geometry}
\geometry{a4paper, margin=2cm}

\title{État d'Avancement du Projet CloneGraph}
\author{Équipe CloneGraph}
\date{\today}

\begin{document}

\maketitle

\section{Introduction}

Le projet CloneGraph vise à développer un détecteur de clones de code multi-langages utilisant des techniques avancées d'analyse de code, incluant les arbres de syntaxe abstraite (AST), les embeddings Graph2Vec et CodeBERT, ainsi que le calcul de similarité distribué avec Apache Spark.

\section{Rappel du Cahier des Charges}

Le cahier des charges spécifie la création d'un système capable de :
\begin{itemize}
\item Analyser du code en Python, Java et C++
\item Générer des AST à l'aide de Tree-sitter
\item Produire des embeddings via Graph2Vec (CPU) et CodeBERT (GPU)
\item Calculer les similarités avec Spark
\item Fournir une interface utilisateur avec Streamlit
\end{itemize}

\section{Travaux Réalisés à ce Jour}

À ce jour, nous avons implémenté :
\begin{itemize}
\item Le module d'extraction et de nettoyage du code
\item La génération d'AST pour les trois langages
\item Les embeddings Graph2Vec sur CPU
\item Le pipeline Spark pour la similarité
\item L'interface Streamlit de base
\item Les notebooks pour CodeBERT et GraphCodeBERT
\end{itemize}

Le taux d'avancement est estimé à 80\%.

\section{Architecture Détaillée du Système}

L'architecture comprend :
\begin{enumerate}
\item Extraction : Chargement et nettoyage du code
\item Génération AST : Utilisation de Tree-sitter
\item Embeddings : Graph2Vec sur CPU, CodeBERT sur GPU
\item Pipeline Spark : Calcul distribué de similarité
\item Détection : Identification des clones
\item Interface : Visualisation avec Streamlit
\end{enumerate}

\begin{figure}[h]
\centering
\includegraphics[width=0.8\textwidth]{architecture_diagram.png}
\caption{Diagramme d'architecture}
\end{figure}

\section{Modules Déjà Implémentés}

\begin{tabular}{|l|l|}
\hline
Module & Statut \\
\hline
Extraction & Implémenté \\
AST Generation & Implémenté \\
Graph2Vec & Implémenté \\
Spark Pipeline & Implémenté \\
Interface & Implémenté \\
\hline
\end{tabular}

\section{Modules en Cours}

\begin{itemize}
\item Optimisation des embeddings CodeBERT
\item Amélioration de la visualisation
\end{itemize}

\section{Difficultés Rencontrées}

\begin{itemize}
\item Intégration de Tree-sitter pour C++
\item Gestion des embeddings haute dimension
\item Performance sur CPU pour les notebooks
\end{itemize}

\section{Plan d'Action pour les Prochaines Semaines}

\begin{itemize}
\item Semaine 1-2 : Finalisation des notebooks GPU
\item Semaine 3 : Tests intégrés
\item Semaine 4 : Documentation et rapport final
\end{itemize}

\begin{figure}[h]
\centering
% Placeholder for Gantt chart
\includegraphics[width=0.8\textwidth]{gantt_chart.png}
\caption{Planning Gantt}
\end{figure}

\section{Conclusion}

Le projet progresse bien malgré les défis techniques. L'équipe est confiante dans la livraison finale.

\end{document}